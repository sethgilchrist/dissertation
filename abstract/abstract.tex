%% This is the abstract for my UBC PhD Dissertation
%% The parent document is called thesis.tex
\chapter{Abstract}
\label{ch:abstract}
Knowledge of proximal femur failure mechanics has a pivotal role to play in predicting who might suffer a hip fracture.
Previous researchers of sideways falls resulting in hip fracture have investigated the roles of several bone parameters such as bone mineral density and morphology, as well as different modelling boundary conditions.
While important advances have been made, current models use constant displacement rates applied at the greater trochanter, which may not allow the bone to respond as it would in a sideways fall.
Proximal femurs have been shown to be sensitive to displacement rate, but impacts like those in a sideways falls have never been examined.
The goal of this thesis was to compare the results of constant displacement rate testing to more biofidelic, inertially driven, impact fall simulation testing and determine how these methods influence specific test outcomes.
In study~1, sub-failure loads were applied to single bones at constant displacement rate, followed by impact loading in the fall simulator.
Stiffnesses, energies and strains at a point were compared.
In study~2, the same bones were compared using digital image correlation to examine bone strains on the anterior-superior femoral neck.
In study~3, two cohorts of bones were loaded to failure, one at constant displacement rate and the other in the impact fall simulator.
Initial failure locations, fracture patterns, stiffnesses and energies to failure were compared.
The results of study~1 indicate that the behaviours of the bones were not affected by the change in loading.
In study~2, I found that individual specimen strain maps were sensitive to the change in loading parameters; however, pooling the data from all the specimens yielded no statistical difference.
In study~3, I discovered that final fracture patterns were different, but initial failure locations, stiffnesses and energies to fracture were not.
Additionally, femurs tested in the fall simulator did not show significant viscoelasticity.
These data indicate that sub-failure measures of mechanical bone behaviour are insensitive to changing between constant displacement rate and impact fall simulation testing; however, strain pattern and final fracture behaviours are influence by changing the loading protocol.