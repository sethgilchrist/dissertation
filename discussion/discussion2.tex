%% This is the discussion for my UBC PhD Dissertation
%% The parent document is called thesis.tex

\chapter{Discussion and conclusions}
\label{ch:discussion}
\section{Discussion}
\label{sec:discussion_discussion}
Hip fracture research, in the larger sense, has one overall goal: to prevent fractures.
There are a number of different methods for attaining this goal, from life-style interventions to pharmacology, but before any of these can be tried, an individual's risk of fracture must be estimated.
This is where current knowledge is limited.
The majority of hip fractures occur in individuals who would not have been identified using osteoporosis state, which is the current clinical standard for beginning treatment~\citep{stone_bmd_2003}.
On top of that, there is considerable overlap in \acp{bmd} of those who suffer fracture and those who do not~\citep{greenspan_fall_1994}.
From these data, it appears that the current understanding of hip fracture risk based primarily on osteoporosis state and \ac{bmd} does not give sufficient sensitivity to help clinicians reduce fracture rates below their current levels.

One way to increase the screening sensitivity might be through improved identification of physical parameters of the proximal femur that predispose a person to fracture.
In the past, gross geometric properties of the proximal femur have been proposed as supplemental data to osteoporosis state to increase the rates of identification.
In the more recent past, the \ac{frax} score was developed which included medical and family history that has been linked to hip fracture through epidemiological studies.
While gross anatomy of the proximal femur showed promise~\citep{faulkner_simple_1993}, it has never been generally adopted into clinical use, and while there are considerable advantages to the use of the \ac{frax} score (namely, pre-screening without using expensive and potentially dangerous \ac{dxa} scanning~\citep{kanis_frax_2009}), the sensitivity of \ac{frax} is only slightly better than \ac{bmd} and age alone~\citep{van_den_bergh_assessment_2010}, and the specificity of \ac{frax} is highly dependent on the selection of risk bounds~\citep{leslie_fracture_2011}.

The fact that \ac{bmd} accounts for only a portion of those that suffer hip fractures leads us to believe that there may be other important features of the bone which could be used by clinicians to screen for those at risk of a fracture.
A number of studies have indicated that the configuration of the bone itself could be important in determining hip fracture risk.
Cortical bone has been seen to become thinner and trabecularized with age, which could result in a decrease in strength~\citep{crabtree_intracapsular_2001, blain_cortical_2008, mayhew_relation_2005}, and cancellous bone has been seen to have a different morphology in fracture patients than in controls~\cite{manske_cortical_2008, blain_cortical_2008}.
While these data indicate that changes on a bone material level may in part be responsible for fracture susceptibility, the hip fracture research community has not been able to definitely identify what features or configurations are culpable for this change in fracture risk.

Hip fracture modelling, both experimental and computational, has been carried out (or validated) using proximal femurs loaded in materials testing machines (see Table~\ref{tab:model_results} on page~\pageref{tab:model_results} for experimental references).
These machines apply forces by applying a defined velocity at the greater trochanter of the femur.
The selection of the velocity has been shown to be important~\citep{courtney_effects_1994, weber_proximal_1992}, but the actual value that should be used on a specimen specific basis has never been determined.
It is possible that the use of a prescribed displacement rate introduces artefacts that may influence how fractures are initiated, how they propagate, and what features influence them.
If this is indeed the case, the previous data must be evaluated with this in mind and conclusions should be limited to those that are appropriate for the boundary conditions used.
This work was intended to address this final point -- what is the effect of a constant velocity at the greater trochanter and what parameters are influenced by this boundary condition?
Knowledge of which parameters are valid using this boundary condition and what parameters require more biofidelic testing will allow researchers to design and develop experiments and validations to address the specific questions they have.
Additionally, previous research can be reinterpreted and applied in a way that is consistent with the validity of its output variables.

The objective of the research, outlined in~\S\ref{sec:intro_goals}, was to determine if current test methods capture the force-displacement, strain, failure and fracture behaviours of the proximal femur in a fall to the side.
This objective was met by comparing and contrasting three aspects of the proximal femur when the fall was modelled using either constant displacement rate tests or impact fall simulation utilizing a lumped parameter model of the human body.
The aspects that were examined were the
\begin{inparaenum}[(i)]
\item mechanical response as characterized by force and displacement and their integral (energy) and derivative (stiffness) measures;
\item failure response as characterized by maximum force, initial cortical failure location and final fracture pattern; and
\item surface strain.
\end{inparaenum}
Comparisons showed that although the quasi-static, constant displacement rate and impact fall simulations methods were equivalent in many cases, there were also differences for some specific outcome variables.

Quasi-static, constant displacement rate testing used in many previous fracture protocols~\citep{backman_proximal_1957, beckmann_femoroplasty--augmentation_2007, boehm_prediction_2008, courtney_age-related_1995, eckstein_reproducibility_2004, heini_femoroplasty-augmentation_2004, keyak_relationships_2000, lochmuller_mechanical_2002, lotz_use_1990} was hypothesized to be a limitation of these experiments.
While there is evidence that increased displacement rate significantly changes the response of the bone~\citep{courtney_effects_1994, weber_proximal_1992}, to our knowledge no research has been conducted that confirmed if this change is indeed representative of the situation \textit{in-vivo}, or if it is an artefact of the increased (but still constant) displacement rate.

The roots of this uncertainty are in the complexity of bone as a material, and the complex arrangement of the bone in structures like the proximal femur.
It is known that bone is viscoelastic~\citep{carter_compressive_1977, linde_mechanical_1991, mcelhaney_dynamic_1966, crowninshield_response_1974} and there is evidence that the strength of the viscoelastic response is influenced by the presence of bone marrow~\citep{carter_compressive_1977}, however, this relationship is not well understood and anticipating its influence in mechanical testing is difficult.
On top of this, bone properties are anisotropic and in homogeneous~\citep{keaveny_biomechanics_2001}, with values and principal directions that vary continuously within the bone structure~\citep{ascenzi_variation_2011, nazarian_densitometric_2007}.
Knowing how the specific arrangement of the bone in a given specimen will be influenced by a given boundary condition is currently outside the scope of our understanding.
We believe that the best way to address these challenges is to allow the proximal femur freedom to respond to an inertially driven system, similar to a fall.
This allows changes in bone material properties and configurations of bone in the specimen to influence the applied loading profile throughout the failure process.
The bone then determines the force and displacement time history during a test, in a recursive way.

In order to grant these freedoms and answer the research questions in this thesis, we designed an inertially driven impact fall simulator that utilized a lumped parameter model of the human body to reproduce the loading in a fall to the side.
This work was detailed in Chapter~\ref{ch:fall_sim_design}.
The advantage of a fall simulator was that it allowed the bone to respond as a compliant, potentially viscous member in a spring mass system and influence the overall behaviour of the system, in the aforementioned recursive way.
This technique removes artificial loading constraints that may influence the behaviour of the bone leading up to and at fracture.
The method was limited in its ability to model a specimen specific fall because the parameters of the lumped parameter model were fixed, \ac{ie}, the stiffness of the spring, mass of the body and pelvis, and thickness of the tissue over the greater trochanter could not be modified based on knowledge of the donor's age, \ac{bmd} or body habitus.
This limitation could be addressed in future experiments, but the background knowledge required to make informed changes to these parameters currently does not exist.
Additionally, increasing the biofidelity of an experiment will not always increase the power of the experiment to be able to address specific questions.
While a realistic test is desired, if too many variables are adjusted between experiments, the statistical burden of proof can become large and make it difficult to obtain the power required to test a hypothesis in a meaningful way.
It is therefore the responsibility of the researcher to identify the important parameters which drive the outcome under investigation and find realistic values for the other parameters.
We believe that the advantage of reduced constraint of the loading rate on the response of the proximal femur during testing outweighs the potential limitation of a single body being simulated for all specimens.

The question relating to the mechanical response in fall simulator testing \ac{vs} materials testing machine testing was addressed using two tests and is contained in Chapter~\ref{ch:behave_fail} in which one set of femurs was tested in two ways.
This experiment was used to determine if the mechanical behaviours of the proximal femurs were affected by the change in loading technique and hypothesized that there would be changes in stiffness, strain at a point and energy to the failure.
We found that there was no difference in the behaviours of the bones in terms of any of our mechanical metrics.
There were no statistically significant differences in the stiffness, energy or strain on the anterior superior femoral neck if the proximal femur is loaded using quasi-static, constant displacement rate or impact fall simulation.
These data indicate that if the outcomes of experimental testing or \ac{fe} validation are sub-failure force-displacement response, then mechanical testing in a materials testing machine is adequate.

We also compared the results of this test to those of \citet{courtney_effects_1994}, who had identified a strong viscoelastic response when displacement rate at the greater trochanter was changed from 2~to 100~\ac{mm}/\ac{s}.
Our results indicate that the viscoelastic response in the fall simulator was much lower than that observed by \citet{courtney_effects_1994}, even though our average displacement rate was nearly the same as the 100~\ac{mm}/\ac{s} rate used by \citet{courtney_effects_1994}.
This indicates that artificially imposing a high displacement rate at the greater trochanter affects outcomes and may not represent the true behaviour of a specimen in a fall to the side.

The question of whether failure initiation and fracture patterns would change due to loading technique was discussed in Chapter~\ref{ch:fracture}, and addressed by hypothesizing that there would be different initial failure locations and final fracture patterns in the constant displacement rate and impact fall simulation testing.
The initial failure locations were identified using high-speed video of the tests and final fracture types were classified using a clinical classification system by an orthopaedic surgeon.
The initial failure locations were not significantly different between the two tests, but the final fracture types were.
Additionally, in the fall simulation tests, the initial failure locations were not predictive of the final fracture types, whereas in the quasi-static tests they were.

These data, combined with the data in the previous paragraphs indicating that the mechanical behaviours are the same for sub-failure loading, indicate that the sub-failure response of the proximal femur is relatively insensitive to the type of loading used.
However, once failure begins and there is a re-definition of the primary load path, the type of testing used becomes important.
The fact that the initial failure observed using high speed video occurred at nearly the same instant as the force-displacement identified yield indicates that the primary load path, regardless of test method, in the sub-failure regime, is through the cortex of the bone.
Once the failure begins, the primary load path changes and observations of the exterior of the bone are no longer sufficient to determine its course.

The final tests conducted were to determine if the sub-failure strains on the surface of the proximal femur were different between quasi-static, constant displacement rate and impact fall simulation tests and is detailed in Chapter~\ref{ch:fracture}.
The results of these tests indicate that the surface strains were different in any given specimen, but taken as a whole (\ac{ie}, pooling all the data) showed that there was no systematic difference between the two test methods.
The majority of the specimens showed a shift in strain levels, either to higher or lower strain, when tested in fall simulation \ac{vs} quasi-static testing.
In these cases, the patters were similar, but there was a ``DC" offset in magnitude.
Some specimens showed a redistribution of strain on the bone cortex, however the magnitude of the average strain was not significantly higher in one test or the other.

These data seem counter to the \textit{no difference in sub-failure mechanical behaviour} result that was identified in Chapter~\ref{ch:behave_fail}.
The reason for this may be due to lack of statistical power to determine differences in loading behaviours in the previous tests.
Those tests had only five data points per specimen, in contrast to the \ac{dic} strain measurements which obtains full strain fields containing hundreds of data points per specimen.
However, it is important to note that the differences observed in the \ac{dic} results were statistically significant, but may not be physically significant.
The average difference in strain was only about 7\% of the strain value, a result that my be due to variations in the experimental setup.
Regardless, the measured differences are real, and based on our data indicating a diverging behaviour after initial failure, the affect on strain would likely increase at higher loads (remember that the response is being compared at 50\% of the \ac{abmd} predicted failure load).
It is therefore our recommendation that if changes in surface strain in a specimen-specific manner is the desired outcome, fall simulation be employed.
However, if the outcome of a series of tests is a generalized behaviour of strain that will be averaged over a cohort, then either quasi-static, constant displacement rate or fall simulation may be employed.

It is important to note that two specimens that were tested in the fall simulator failed but did not fracture.
Additionally, the failures occurred in the same place and had the same morphologies: cracks running along the superior femoral neck, originating in the greater trochanter and terminating in the lateral neck.
Due to their location and size it is unlikely that these failures would have been detected using clinical tools like anterior-posterior x-rays or \ac{ct} scanning.
It is not impossible that failures like these are common after a fall, and go undetected in the population.
Given that our results suggest that cancellous bone mechanics may influence fracture behaviour in a fall after the cortex has failed, understanding the loading paths in these specimens might help inform researchers what made them resistant to fracture.

\ac{fe} analysis, which is on-going in our \ac{fe} collaborator's lab at ETH Z\"urich, is one technique that has specific advantages for detailed exploration of the behaviour of a specific specimen.
The non-destructive nature and ability to examine many different outcomes as a function of time can illuminate events that are difficult to observe experimentally leading up to failure.
That said, the secret to these specimens may lie in the behaviour of their cancellous bone which, as I have pointed out, has not been validated in any \ac{fe} studies.
The work on \acf{dvc} presented in Appendix~\ref{ch:dvc} of this thesis may therefore be critical to understanding how specimens that are resilient to fracture behave.
It appears from the quasi-static and fall simulation work that cancellous bone behaviour is critically important after initial cortical failure, and therefore accurate modelling is crucial for understanding the behaviour of these specimens.
The \ac{dvc} technique and tools developed herein could be used to validate quasi-static strain behaviour of cancellous bone and allow \ac{fe} researchers to advance their understand the role of cancellous bone.
While the nature of the \ac{dvc} technique requires that tests be conducted in a quasi-static manner, the validation is no less important since initial failure location (which may be the critical feature of these specimens) can be predicted in quasi-static tests.

Overall, the results of these tests indicate that different tests are appropriate for different outcomes in proximal femur fracture research (Table~\ref{tab:tests_and_outcomes}).
In general, sub-failure testing that does not rely on surface strain profiles can be conducted using either quasi-static, constant displacement rates or fall simulation.
If post failure mechanics (such as fracture mechanics) or detailed surface strain maps are to be considered, then fall simulation is the best choice.
The implication of the changing post failure mechanics is that dynamics that cannot be observed using high-speed video of tests are important, in which case it is likely that observation of the cancellous bone mechanics and changing loading paths may be informative.

\begin{table}
\centering
\caption[Required tests for outcomes]{Required tests for given outcome variables in mechanical testing and \acs{fe} validation}
\label{tab:tests_and_outcomes}
\begin{tabularx}{0.75\textwidth}{l >{\centering\arraybackslash}X}
\toprule
Outcome Measure & Testing Options \\
\midrule
Sub-failure mechanical behaviour & Quasi-static, constant displacement rate or impact fall simulation \\[1EX]
Initial fracture location & Quasi-static, constant displacement rate or impact fall simulation \\[1EX]
Final fracture pattern & Impact fall simulation \\[1EX]
Specimen specific surface strain & Impact fall simulation \\[1EX]
Cohort surface strain & Quasi-static, constant displacement rate or impact fall simulation \\
\bottomrule 
\end{tabularx}
\end{table}

\section{Conclusion}
\label{sec:discussion_conclusion}
This research addresses one of the most basic questions in hip fracture testing: how biofidelic do our tests need to be?
It was conducted using a first-of-its-kind impact fall simulator and pioneered using \ac{dic} to assess strain on femoral bone.
The experiments resulted in a framework for identifying what tests are appropriate for a given outcome variable and showed that full field strain measurement on the surface of the bone can be used to detect small differences in response to a given loading.
Specimens which did not fracture under an impact loading were identified, which could lead to new insights into the mechanics of resilient bones, and the source of that resilience.

We found that sub-failure mechanical behaviours of the proximal femur are insensitive to the vastly different methods we used to apply loads representing a fall to the side.
Additionally, surface strains of individual specimens were different, but overall strain behaviour of all the specimens were consistent between the two testing methods.
Finally, initial failure locations were not different between the two testing methods, but final fracture patterns were.

These results indicated that there is need for careful consideration of the outcomes of a given test before selecting a test method.
Additionally should a biofidelic, impact test be used, the tools developed herein can be applied to detect small differences in behaviour as well as identify those that are resistant to hip fracture.
This knowledge will allow fracture researchers to make better decisions in design, interpretation and application of tests to identify those at risk of fracture and hopefully prevent the onset of the devastating cascade that can follow.

\section{Future work}
\label{sec:discussion_future}
As with many studies, the work presented in this thesis has illuminated new holes in the knowledge, and helped determine which previously known deficiencies require further investigation.
Recommendations for future work that will, in my opinion, assist researchers in their efforts to understand proximal femur fracture are given below.

\textit{Increase tissue thickness to identify resilient specimens.}
In the current study, two specimens were impacted and did not fracture (Chapter~\ref{ch:behave_fail}).
In real world falls, only about 5\% of falls result in fracture and one of the likely reasons for the difference between our fracture rates and those in the real world is that the current test used a lower bound on soft tissue thickness over the greater trochanter.
In the proposed study, the soft tissue thickness would be increased to one \acl{sd} above the average for fracture cases.
The results from previous researchers have indicated that this increase in soft tissue thickness would likely result in a lower peak force by about 1700~\ac{n}~\citep{robinovitch_force_1995, nielson_trochanteric_2009} and generate more differentiation between strong and weak specimens.
The identification of strong and weak specimens could allow for analysis of cancellous and cortical bone parameters which may indicate why some bones are more resilient to fracture than others.

\textit{Regional \ac{hrpqct} examination of fracture initiation location and non-fracture specimens.}
The location of fracture initiation in the proximal femur is neither regular, nor 100\% predictive of the final fracture type (Chapter~\ref{ch:fracture}).
In this study, specimens that failed would have regional analyses of the cortical and cancellous bone performed on \ac{hrpqct} scans that were acquired before fracture testing.
Additionally, bones that did not fracture in fall simulation testing would have critical regions like the superio-lateral femoral neck and anterior intertrochanteric region evaluated.
\ac{hrpqct} outcomes would be compared between the fracture and non-fracture groups to identify those that are predictive of initial fracture location.

\textit{Comparison of fall simulation generated fractures to clinical fractures.}
After fracture in the current fall simulator, the drop tower gantry continues moving, compressing the remaining bone fragments.
This creates many secondary and tertiary fracture lines and makes post fracture x-ray impossible (Chapter~\ref{ch:fracture}).
In this study, a method, such as stop-blocks, would be fitted in to the fall simulator to prevent further compression of the specimen after a certain compression level, likely in the range of 10~\ac{mm}, had been attained.
Specimens from this group would be imaged using planar x-ray and read by an orthopaedic surgeon or radiologist for comparison to clinical fracture groups.

\textit{Inclusion of an acetabular mass to create compaction injuries.}
In the current fall simulator, the head of the femur has no constraint, and is often separated fully from the femoral neck after fracture (Chapters~\ref{ch:fall_sim_design} and~\ref{ch:behave_fail}).
In a real fall, the head is contained in the acetabulum which is immobilized by the mass of the body.
Inertial constraint of the femoral head, as would be the case in a real fall to the side, may increase the number of intrecapsular fractures and impactions of the femoral neck, increasing the biofidelity of the apparatus.
Continued improvement of the apparatus biofidelity could lead to more clinically relevant fracture types and potentially more relevant fracture locations and patters.