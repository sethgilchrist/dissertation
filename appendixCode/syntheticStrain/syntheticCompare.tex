\begin{lstlisting}
//      C++ Source.cpp
//      
//      Copyright 2010 Seth Gilchrist <seth@mech.ubc.ca>
//      
//      This program is free software; you can redistribute it and/or modify
//      it under the terms of the GNU General Public License as published by
//      the Free Software Foundation; either version 2 of the License, or
//      (at your option) any later version.
//      
//      This program is distributed in the hope that it will be useful,
//      but WITHOUT ANY WARRANTY; without even the implied warranty of
//      MERCHANTABILITY or FITNESS FOR A PARTICULAR PURPOSE.  See the
//      GNU General Public License for more details.
//      
//      You should have received a copy of the GNU General Public License
//      along with this program; if not, write to the Free Software
//      Foundation, Inc., 51 Franklin Street, Fifth Floor, Boston,
//      MA 02110-1301, USA.


#include <iostream>

#include "itkImage.h"
#include "itkVector.h"
#include "itkImageFileReader.h"

#include "vtkUnstructuredGridReader.h"
#include "vtkUnstructuredGrid.h"
#include "vtkDoubleArray.h"
#include "vtkCell.h"
#include "vtkUnstructuredGridWriter.h"
#include "vtkSmartPointer.h"
#include "vtkPointData.h"
#include "vtkCellData.h"
#include "vtkCellDerivatives.h"
#include "vtkCellDataToPointData.h"
#include "vtkMath.h"

#include "itkImageRegionConstIterator.h"
#include "itkImageRegionIterator.h"
#include "itkLinearInterpolateImageFunction.h"


class CompareResults {
	// Mesh Image Types
	typedef vtkUnstructuredGrid						MeshImageType;
	typedef vtkDoubleArray							MeshPixelType;
	typedef vtkPointSet								MeshPointType;
	typedef vtkCell									MeshCellType;
	typedef vtkUnstructuredGridReader				MeshReaderType;
	typedef vtkUnstructuredGridWriter				MeshWriterType;
	// Vector Image Types
	typedef double									VectorComponentType;
	typedef itk::Vector< VectorComponentType, 3 >	VectorPixelType;
	typedef itk::Image< VectorPixelType, 3 >		VectorImageType;
	typedef	itk::ImageFileReader< VectorImageType >	VectorImageReaderType;
	// Z Displacement Image Type
	typedef itk::Image< VectorComponentType, 3 >	XDispImageType;
	typedef itk::Image< VectorComponentType, 3 >	YDispImageType;	
	typedef itk::Image< VectorComponentType, 3 >	ZDispImageType;

	
	
	public:
	CompareResults();
	~CompareResults();
	
	void SetInputMeshFileName( std::string fileName );
	void SetOutputMeshFileName( std::string fileName );
	void SetVectorImageFileName( std::string fileName );
	
	void ReadMeshImage();
	void ReadVectorImage();
	void WriteVectorImage();
	
	void SetInputMeshImage( MeshImageType *meshImage );
	void SetInputVectorImage( VectorImageType::Pointer vectorImage );
	void SetOutputMeshImage( MeshImageType *meshImage );
	
	void CreateOutputImage();
	void ComputeDisplacementImages();
	void ComputeMeshDisplacement();
	void ComputeMeshStrain();
	void ComputeMeshPrincipalStrain();
	void ComputeAbsoluteDifference();
	void ComputeAbsoluteStrainDifference();
	void ComputeVoxelNormalizedDifference();
	
	private:
	vtkSmartPointer<MeshImageType>	m_InputMeshImage;
	vtkSmartPointer<MeshImageType>	m_OutputMeshImage;
	vtkSmartPointer<MeshReaderType>	m_MeshReader;
	vtkSmartPointer<MeshWriterType>	m_MeshWriter;
	VectorImageReaderType::Pointer	m_VectorReader;
	VectorImageType::Pointer		m_InputVectorImage;
	ZDispImageType::Pointer			m_XDispImage;
	ZDispImageType::Pointer			m_YDispImage;
	ZDispImageType::Pointer			m_ZDispImage;
	std::string						m_InputMeshFileName;
	std::string						m_OutputMeshFileName;
	std::string						m_InputVectorFileName;
};

CompareResults::CompareResults(){
	m_InputMeshImage		= vtkSmartPointer<MeshImageType>::New();
	m_OutputMeshImage		= vtkSmartPointer<MeshImageType>::New();
	m_InputVectorImage		= VectorImageType::New();
	m_XDispImage			= XDispImageType::New();
	m_YDispImage			= YDispImageType::New();
	m_ZDispImage			= ZDispImageType::New();
	m_MeshReader			= vtkSmartPointer<MeshReaderType>::New();
	m_MeshWriter			= vtkSmartPointer<MeshWriterType>::New();
	m_VectorReader			= VectorImageReaderType::New();
}

CompareResults::~CompareResults(){}

void CompareResults::SetInputMeshFileName( std::string fileName )
{
	if( m_InputMeshFileName != fileName ){
		m_InputMeshFileName = fileName;
	}
}
void CompareResults::SetOutputMeshFileName( std::string fileName )
{
	if( m_OutputMeshFileName != fileName ){
		m_OutputMeshFileName = fileName;
	}
}
void CompareResults::SetVectorImageFileName( std::string fileName )
{
	if(m_InputVectorFileName != fileName ){
		m_InputVectorFileName = fileName;
	}
}

void CompareResults::ReadMeshImage()
{
	m_MeshReader->SetFileName( m_InputMeshFileName.c_str() );
	m_MeshReader->Update();
	CompareResults::SetInputMeshImage( m_MeshReader->GetOutput() );
}
void CompareResults::ReadVectorImage()
{
	m_VectorReader->SetFileName( m_InputVectorFileName.c_str() );
	m_VectorReader->Update();
	CompareResults::SetInputVectorImage( m_VectorReader->GetOutput() );	
}
void CompareResults::WriteVectorImage()
{
	m_MeshWriter->SetFileName( m_OutputMeshFileName.c_str() );
	m_MeshWriter->SetInput( m_OutputMeshImage );
	m_MeshWriter->Update();	
}
void CompareResults::SetOutputMeshImage( MeshImageType *meshImage )
{
	if( m_OutputMeshImage.GetPointer() != meshImage){
		m_OutputMeshImage = meshImage;
	}
}
void CompareResults::SetInputMeshImage( MeshImageType *meshImage )
{
	if( m_InputMeshImage.GetPointer() != meshImage ){
		m_InputMeshImage = meshImage;
	}
}
void CompareResults::SetInputVectorImage( VectorImageType::Pointer vectorImage )
{
	if( m_InputVectorImage.GetPointer() != vectorImage){
		m_InputVectorImage = vectorImage;
	}
}


void CompareResults::CreateOutputImage()
{
	// set the points
	m_OutputMeshImage->SetPoints( m_InputMeshImage->GetPoints() );
	// set the cells
	m_OutputMeshImage->SetCells( m_InputMeshImage->GetCellTypesArray(),
		m_InputMeshImage->GetCellLocationsArray(),
		m_InputMeshImage->GetCells() );

	// create the array for the absolute difference
	vtkSmartPointer<vtkDoubleArray>		absoluteDifference = vtkSmartPointer<vtkDoubleArray>::New();
	absoluteDifference->SetNumberOfComponents( 3 );	
	absoluteDifference->SetNumberOfTuples( m_OutputMeshImage->GetNumberOfPoints() );
	absoluteDifference->SetName( "Absolute Displacement Difference" );

	// create the array for the expected displacement
	vtkSmartPointer<vtkDoubleArray>		displacement = vtkSmartPointer<vtkDoubleArray>::New();
	displacement->SetNumberOfComponents( 3 );	
	displacement->SetNumberOfTuples( m_OutputMeshImage->GetNumberOfPoints() );
	displacement->SetName( "Actual Displacement" );
	
	// create the array for the real value normalized difference
	vtkSmartPointer<vtkDoubleArray>		relativeDifference = vtkSmartPointer<vtkDoubleArray>::New();
	relativeDifference->SetNumberOfComponents( 3 );	
	relativeDifference->SetNumberOfTuples( m_OutputMeshImage->GetNumberOfPoints() );
	relativeDifference->SetName( "Voxel Size Normalized Difference" );
	
	vtkSmartPointer<vtkDoubleArray>		absoluteStrainDifference = vtkSmartPointer<vtkDoubleArray>::New();
	absoluteStrainDifference->SetNumberOfComponents( 3 );
	absoluteStrainDifference->SetNumberOfTuples( m_OutputMeshImage->GetNumberOfPoints() );
	absoluteStrainDifference->SetName( "Absolute Strain Difference" );
	
	// create an array to pass the principal strain calcualted byt the DVC
	vtkSmartPointer<vtkDoubleArray>		calculatedPrinicpalStrains = vtkSmartPointer<vtkDoubleArray>::New();
	calculatedPrinicpalStrains->SetNumberOfComponents( 3 );
	calculatedPrinicpalStrains->SetNumberOfTuples( m_OutputMeshImage->GetNumberOfPoints() );
	calculatedPrinicpalStrains->SetName( "DVC Principal Strains" );
	for (int i = 0; i < m_OutputMeshImage->GetNumberOfPoints(); ++i ){
		double *cStrainPoint0 = m_InputMeshImage->GetPointData()->GetArray("Principal Strain Value 1")->GetTuple( i );
		double *cStrainPoint1 = m_InputMeshImage->GetPointData()->GetArray("Principal Strain Value 2")->GetTuple( i );
		double *cStrainPoint2 = m_InputMeshImage->GetPointData()->GetArray("Principal Strain Value 3")->GetTuple( i );
		double *cStrainPoint = new double[3];
		*cStrainPoint = *cStrainPoint0;
		*(cStrainPoint+1) = *cStrainPoint1;
		*(cStrainPoint+2) = *cStrainPoint2;
		calculatedPrinicpalStrains->SetTuple( i, cStrainPoint );
	}

	// create an array to pass the strain tensors calcualted byt the DVC
	vtkSmartPointer<vtkDoubleArray>		calculatedStrainTensor = vtkSmartPointer<vtkDoubleArray>::New();
	calculatedStrainTensor->SetNumberOfComponents( 9 );
	calculatedStrainTensor->SetNumberOfTuples( m_OutputMeshImage->GetNumberOfPoints() );
	calculatedStrainTensor->SetName( "DVC Strain Tensor" );
	for (int i = 0; i < m_OutputMeshImage->GetNumberOfPoints(); ++i ){	
		double *cStrainTensor = new double[9];
		m_InputMeshImage->GetPointData()->GetArray("Strain")->GetTuple( i, cStrainTensor );
		calculatedStrainTensor->SetTuple( i, cStrainTensor );
	}

	// add the data arrays to the output image
	m_OutputMeshImage->GetPointData()->AddArray( absoluteDifference );
	m_OutputMeshImage->GetPointData()->AddArray( displacement );
	m_OutputMeshImage->GetPointData()->AddArray( relativeDifference );
	m_OutputMeshImage->GetPointData()->AddArray( absoluteStrainDifference );
	m_OutputMeshImage->GetPointData()->AddArray( calculatedPrinicpalStrains );
	m_OutputMeshImage->GetPointData()->AddArray( calculatedStrainTensor );
}
void CompareResults::ComputeDisplacementImages()
{
	// set the size, shape and spacing of the images for interpolation
	m_XDispImage->SetOrigin( m_InputVectorImage->GetOrigin() );
	m_XDispImage->SetSpacing( m_InputVectorImage->GetSpacing() );
	m_XDispImage->SetRegions( m_InputVectorImage->GetLargestPossibleRegion() );
	m_XDispImage->Allocate();
	
	m_YDispImage->SetOrigin( m_InputVectorImage->GetOrigin() );
	m_YDispImage->SetSpacing( m_InputVectorImage->GetSpacing() );
	m_YDispImage->SetRegions( m_InputVectorImage->GetLargestPossibleRegion() );
	m_YDispImage->Allocate();
	
	m_ZDispImage->SetOrigin( m_InputVectorImage->GetOrigin() );
	m_ZDispImage->SetSpacing( m_InputVectorImage->GetSpacing() );
	m_ZDispImage->SetRegions( m_InputVectorImage->GetLargestPossibleRegion() );
	m_ZDispImage->Allocate();
	
	// iterate through the input vector image and extract the components
	typedef itk::ImageRegionConstIterator< VectorImageType >		VectorImageIteratorType;
	VectorImageIteratorType		vIt( m_InputVectorImage, m_InputVectorImage->GetLargestPossibleRegion() );
	
	typedef itk::ImageRegionIterator< XDispImageType >				XDispImageIteratorType;
	typedef itk::ImageRegionIterator< YDispImageType >				YDispImageIteratorType;
	typedef itk::ImageRegionIterator< ZDispImageType >				ZDispImageIteratorType;
	XDispImageIteratorType		xIt( m_XDispImage, m_InputVectorImage->GetLargestPossibleRegion() );
	YDispImageIteratorType		yIt( m_YDispImage, m_InputVectorImage->GetLargestPossibleRegion() );
	ZDispImageIteratorType		zIt( m_ZDispImage, m_InputVectorImage->GetLargestPossibleRegion() );
	
	for( vIt.GoToBegin(), xIt.GoToBegin(), yIt.GoToBegin(), zIt.GoToBegin(); !vIt.IsAtEnd(); ++vIt, ++xIt, ++yIt, ++zIt ){
		VectorImageType::PixelType	cVPixel = vIt.Get();
		
		XDispImageType::PixelType	cXPixel = -cVPixel[0]; // the DVC returns the opposite of the warp image.
		YDispImageType::PixelType	cYPixel = -cVPixel[1];
		ZDispImageType::PixelType	cZPixel = -cVPixel[2];
		
		xIt.Set( cXPixel );
		yIt.Set( cYPixel );
		zIt.Set( cZPixel );
	}
}
void CompareResults::ComputeMeshDisplacement()
{
	// define the interpolator that we are going to use
	typedef itk::LinearInterpolateImageFunction< XDispImageType, double >	XInterpolatorType;
	XInterpolatorType::Pointer	xInterpolator = XInterpolatorType::New();
	xInterpolator->SetInputImage( m_XDispImage );
	typedef itk::LinearInterpolateImageFunction< YDispImageType, double >	YInterpolatorType;
	YInterpolatorType::Pointer	yInterpolator = YInterpolatorType::New();
	yInterpolator->SetInputImage( m_YDispImage );
	typedef itk::LinearInterpolateImageFunction< ZDispImageType, double >	ZInterpolatorType;
	ZInterpolatorType::Pointer	zInterpolator = ZInterpolatorType::New();
	zInterpolator->SetInputImage( m_ZDispImage );
	
	unsigned int nPoints = m_OutputMeshImage->GetNumberOfPoints();
	for( unsigned int i = 0; i < nPoints; ++i ){
		ZDispImageType::PointType	cZPt;
		double *cMPt = m_OutputMeshImage->GetPoint( i );
		cZPt[0] = *cMPt; cZPt[1] = *(cMPt+1); cZPt[2] = *(cMPt+2);
		XDispImageType::PixelType xValue = xInterpolator->Evaluate( cZPt );
		YDispImageType::PixelType yValue = yInterpolator->Evaluate( cZPt );
		ZDispImageType::PixelType zValue = zInterpolator->Evaluate( cZPt );
		double value[3];
		value[0] = xValue;
		value[1] = yValue;
		value[2] = zValue;
		
		m_OutputMeshImage->GetPointData()->GetArray( "Actual Displacement" )->SetTuple( i, value );
	}
}
void CompareResults::ComputeMeshStrain()
{
	// calculate derivatives
	vtkCellDerivatives *derivativeCalculator = vtkCellDerivatives::New();
	derivativeCalculator->SetInput( this->m_OutputMeshImage );
	derivativeCalculator->SetTensorModeToComputeStrain();
	derivativeCalculator->SetVectorModeToComputeGradient();
	
	derivativeCalculator->SetInputArrayToProcess(1,0,0,0,"Actual Displacement");
	derivativeCalculator->Update();
	this->SetOutputMeshImage( derivativeCalculator->GetUnstructuredGridOutput() );
		
	// move the data from cells to nodes
	vtkCellDataToPointData	*dataTransformer = vtkCellDataToPointData::New();
	dataTransformer->SetInput( this->m_OutputMeshImage );
	dataTransformer->PassCellDataOn();
	dataTransformer->SetInputArrayToProcess(0,0,1,1,"Strain");
	dataTransformer->Update();
	
	this->SetOutputMeshImage( dataTransformer->GetUnstructuredGridOutput() );
}
void CompareResults::ComputeMeshPrincipalStrain()
{
	vtkSmartPointer<vtkMath>		mathAlgorithm = vtkSmartPointer<vtkMath>::New();
	// First find the principal strains for the point data
	vtkIdType nPoints = this->m_OutputMeshImage->GetPointData()->GetArray("Strain")->GetNumberOfTuples();
	
	typedef	vtkSmartPointer<vtkDoubleArray>		DataImagePixelPointer;
	
	// create the eigenvector containers
	DataImagePixelPointer V0 = DataImagePixelPointer::New();
	V0->SetNumberOfComponents(3);
	V0->SetName("Actual Principal Strain Vector 1");
	DataImagePixelPointer V1 = DataImagePixelPointer::New();
	V1->SetNumberOfComponents(3);
	V1->SetName("Actual Principal Strain Vector 2");	
	DataImagePixelPointer V2 = DataImagePixelPointer::New();
	V2->SetNumberOfComponents(3);
	V2->SetName("Actual Principal Strain Vector 3");

	// create the eigenvalue containers
	DataImagePixelPointer	val0 = DataImagePixelPointer::New();
	val0->SetNumberOfComponents(1);
	val0->SetName("Actual Principal Strain Value 1");
	DataImagePixelPointer	val1 = DataImagePixelPointer::New();
	val1->SetNumberOfComponents(1);
	val1->SetName("Actual Principal Strain Value 2");
	DataImagePixelPointer	val2 = DataImagePixelPointer::New();
	val2->SetNumberOfComponents(1);
	val2->SetName("Actual Principal Strain Value 3");
	
	
	for ( unsigned int i = 0; i < nPoints; ++i ){
		// Get the point tensor
		double *cTensorRaw = this->m_OutputMeshImage->GetPointData()->GetArray("Strain")->GetTuple( i );
		// reform tensor into an appropreate array
		double *cTensor[3];
		double cT0[3];
		double cT1[3];
		double cT2[3];

		cTensor[0] = cT0; cTensor[1] = cT1; cTensor[2] = cT2;

		cT0[0] = *cTensorRaw;
		cT0[1] = *(cTensorRaw+1);
		cT0[2] = *(cTensorRaw+2);	
		cT1[0] = *(cTensorRaw+3);
		cT1[1] = *(cTensorRaw+4);
		cT1[2] = *(cTensorRaw+5);
		cT2[0] = *(cTensorRaw+6);
		cT2[1] = *(cTensorRaw+7);
		cT2[2] = *(cTensorRaw+8);
				
		// Get the eigen-values and -vectors
		double eigenValues[3]; // The Eigenvalues
		double *eigenVectors[3]; // vector of pointers to eigenvectors
		double v0[3]; // eigen vector 1
		double v1[3]; // eigen vector 2
		double v2[3]; // eigen vector 3
		eigenVectors[0] = v0; eigenVectors[1] = v1; eigenVectors[2] = v2;
		
		// perform the calculation of the principal strains
		mathAlgorithm->Jacobi(cTensor,eigenValues,eigenVectors);
		
		// save the values and vectors into the data image
		V0->InsertNextTuple( eigenVectors[0] );
		val0->InsertNextTuple( &eigenValues[0] );
		V1->InsertNextTuple( eigenVectors[1] );
		val1->InsertNextTuple( &eigenValues[1] );
		V2->InsertNextTuple( eigenVectors[2] );
		val2->InsertNextTuple( &eigenValues[2] );
	}
	
	this->m_OutputMeshImage->GetPointData()->AddArray( V0 );
	this->m_OutputMeshImage->GetPointData()->AddArray( val0 );
	this->m_OutputMeshImage->GetPointData()->AddArray( V1 );
	this->m_OutputMeshImage->GetPointData()->AddArray( val1 );
	this->m_OutputMeshImage->GetPointData()->AddArray( V2 );
	this->m_OutputMeshImage->GetPointData()->AddArray( val2 );
	
	// next find the principal strains for the cell data
	nPoints = this->m_OutputMeshImage->GetCellData()->GetArray("Strain")->GetNumberOfTuples();
	
	// create the eigenvector containers
	V0 = DataImagePixelPointer::New();
	V0->SetNumberOfComponents(3);
	V0->SetName("Actual Principal Strain Vector 1");
	V1 = DataImagePixelPointer::New();
	V1->SetNumberOfComponents(3);
	V1->SetName("Actual Principal Strain Vector 2");	
	V2 = DataImagePixelPointer::New();
	V2->SetNumberOfComponents(3);
	V2->SetName("Actual Principal Strain Vector 3");

	// create the eigenvalue containers
	val0 = DataImagePixelPointer::New();
	val0->SetNumberOfComponents(1);
	val0->SetName("Actual Principal Strain Value 1");
	val1 = DataImagePixelPointer::New();
	val1->SetNumberOfComponents(1);
	val1->SetName("Actual Principal Strain Value 2");
	val2 = DataImagePixelPointer::New();
	val2->SetNumberOfComponents(1);
	val2->SetName("Actual Principal Strain Value 3");
	
	
	for ( unsigned int i = 0; i < nPoints; ++i ){
		// Get the point tensor
		double *cTensorRaw = this->m_OutputMeshImage->GetCellData()->GetArray("Strain")->GetTuple( i );
		// reform tensor into an appropreate array
		double *cTensor[3];
		double cT0[3];
		double cT1[3];
		double cT2[3];

		cTensor[0] = cT0; cTensor[1] = cT1; cTensor[2] = cT2;

		cT0[0] = *cTensorRaw;
		cT0[1] = *(cTensorRaw+1);
		cT0[2] = *(cTensorRaw+2);	
		cT1[0] = *(cTensorRaw+3);
		cT1[1] = *(cTensorRaw+4);
		cT1[2] = *(cTensorRaw+5);
		cT2[0] = *(cTensorRaw+6);
		cT2[1] = *(cTensorRaw+7);
		cT2[2] = *(cTensorRaw+8);
				
		// Get the eigen-values and -vectors
		double eigenValues[3]; // The Eigenvalues
		double *eigenVectors[3]; // vector of pointers to eigenvectors
		double v0[3]; // eigen vector 1
		double v1[3]; // eigen vector 2
		double v2[3]; // eigen vector 3
		eigenVectors[0] = v0; eigenVectors[1] = v1; eigenVectors[2] = v2;
		
		// perform the calculation of the principal strains
		mathAlgorithm->Jacobi(cTensor,eigenValues,eigenVectors);
		
		// save the values and vectors into the data image
		V0->InsertNextTuple( eigenVectors[0] );
		val0->InsertNextTuple( &eigenValues[0] );
		V1->InsertNextTuple( eigenVectors[1] );
		val1->InsertNextTuple( &eigenValues[1] );
		V2->InsertNextTuple( eigenVectors[2] );
		val2->InsertNextTuple( &eigenValues[2] );
	}
	
	this->m_OutputMeshImage->GetCellData()->AddArray( V0 );
	this->m_OutputMeshImage->GetCellData()->AddArray( val0 );
	this->m_OutputMeshImage->GetCellData()->AddArray( V1 );
	this->m_OutputMeshImage->GetCellData()->AddArray( val1 );
	this->m_OutputMeshImage->GetCellData()->AddArray( V2 );
	this->m_OutputMeshImage->GetCellData()->AddArray( val2 );
}
void CompareResults::ComputeAbsoluteDifference()
{
	unsigned int nPoints = m_OutputMeshImage->GetNumberOfPoints();
	for( unsigned int i = 0; i < nPoints; ++i){
		double *aDisp = m_OutputMeshImage->GetPointData()->GetArray( "Actual Displacement" )->GetTuple( i ); // actual difference
		double *cDisp = m_InputMeshImage->GetPointData()->GetArray( "Displacement" )->GetTuple( i ); // calculated difference
		double aDiff[3]; 
		aDiff[0] = std::abs( *cDisp - *aDisp ); 
		aDiff[1] = std::abs( *(cDisp+1) - *(aDisp+1) ); 
		aDiff[2] = std::abs( *(cDisp+2) - *(aDisp+2) );
		m_OutputMeshImage->GetPointData()->GetArray( "Absolute Displacement Difference" )->SetTuple( i, aDiff );
	}
}
void CompareResults::ComputeAbsoluteStrainDifference()
{
	unsigned int nPoints = m_OutputMeshImage->GetNumberOfPoints();
	for( unsigned int i = 0; i < nPoints; ++i){
		
		double *aStrain1 = m_OutputMeshImage->GetPointData()->GetArray( "Actual Principal Strain Value 1" )->GetTuple( i ); // actual strain 1
		double *cStrain1 = m_InputMeshImage->GetPointData()->GetArray( "Principal Strain Value 1" )->GetTuple( i ); // calculated strain 1
		
		double *aStrain2 = m_OutputMeshImage->GetPointData()->GetArray( "Actual Principal Strain Value 2" )->GetTuple( i ); // actual strain 2
		double *cStrain2 = m_InputMeshImage->GetPointData()->GetArray( "Principal Strain Value 2" )->GetTuple( i ); // calculated strain 2
		
		double *aStrain3 = m_OutputMeshImage->GetPointData()->GetArray( "Actual Principal Strain Value 3" )->GetTuple( i ); // actual strain 3
		double *cStrain3 = m_InputMeshImage->GetPointData()->GetArray( "Principal Strain Value 3" )->GetTuple( i ); // calculated strain 3
		
		double aDiff[3]; 
		aDiff[0] = std::abs( *cStrain1 - *aStrain1 );
		aDiff[1] = std::abs( *cStrain2 - *aStrain2 );
		aDiff[2] = std::abs( *cStrain3 - *aStrain3 );
		m_OutputMeshImage->GetPointData()->GetArray( "Absolute Strain Difference" )->SetTuple( i, aDiff );
	}
}
void CompareResults::ComputeVoxelNormalizedDifference()
{
	unsigned int nPoints = m_OutputMeshImage->GetNumberOfPoints();
	for( unsigned int i = 0; i <nPoints; ++i ){
		double *cDiff = m_OutputMeshImage->GetPointData()->GetArray( "Absolute Displacement Difference" )->GetTuple( i ); // calculated difference
		VectorImageType::SpacingType voxelSpacing = m_InputVectorImage->GetSpacing();
		double nDiff[3]; // the normalized difference 
		nDiff[0] = std::abs(*cDiff / voxelSpacing[0]);
		nDiff[1] = std::abs(*(cDiff+1) / voxelSpacing[1]);
		nDiff[2] = std::abs(*(cDiff+2) / voxelSpacing[2]);
		m_OutputMeshImage->GetPointData()->GetArray( "Voxel Size Normalized Difference" )->SetTuple( i, nDiff );
	}
}


int main(int argc, char** argv)
{
	if( argc < 4 || argc > 4){ // if the the inputs are wrong, abort
		std::cout<<"Wrong number of inputs."<<std::endl;
		std::cout<<"Usage:"<<std::endl<<argv[0]<<" [InputMeshImage] [VectorImage] [OutputMeshImage]"<<std::endl;
		std::cout<<"InputMeshImage = Mesh image for analysis"<<std::endl;
		std::cout<<"VectorImage = Vector image to compare the mesh image to"<<std::endl;
		std::cout<<"OutputMeshImage = Output mesh image"<<std::endl;
		std::cout<<"Aborting."<<std::endl;
		return EXIT_FAILURE;
	}

	CompareResults *comp = new CompareResults;
	
	std::string meshInput = argv[1];
	std::string vectorInput = argv[2];
	std::string meshOutput = argv[3];
	comp->SetInputMeshFileName( meshInput );
	comp->SetVectorImageFileName( vectorInput );
	comp->SetOutputMeshFileName( meshOutput );
	
	std::cout<<std::endl<<"Reading the input mesh file"<<std::endl;
	comp->ReadMeshImage();
	std::cout<<"Reading the input vector image"<<std::endl;
	comp->ReadVectorImage();
	std::cout<<"Creating the output image"<<std::endl;
	comp->CreateOutputImage();
	std::cout<<"Creating the displacement images for interpolation"<<std::endl;
	comp->ComputeDisplacementImages();
	std::cout<<"Interpolating the displacement data"<<std::endl;
	comp->ComputeMeshDisplacement();
	std::cout<<"Calculating actual strains"<<std::endl;
	comp->ComputeMeshStrain();
	std::cout<<"Calculating actual principal strains"<<std::endl;
	comp->ComputeMeshPrincipalStrain();
	std::cout<<"Calculating the absolute difference between the actual and calculated displacements"<<std::endl;
	comp->ComputeAbsoluteDifference();
	std::cout<<"Calculating the voxel spacing normalized difference between actual and calculated displacement values"<<std::endl;
	comp->ComputeVoxelNormalizedDifference();
	std::cout<<"Calculating the absolute difference between the actual and calculated strains"<<std::endl;
	comp->ComputeAbsoluteStrainDifference();
	
	std::cout<<"Writing the output to: "<<meshOutput<<std::endl;
	comp->WriteVectorImage();
	std::cout<<"Evaluation Complete"<<std::endl<<std::endl;
	return 0;
}

\end{lstlisting}