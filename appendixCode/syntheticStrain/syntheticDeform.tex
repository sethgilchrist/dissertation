\begin{lstlisting}
//      SyntheticImageStrain.cxx
//      
//      Copyright 2010 Seth Gilchrist <seth@mech.ubc.ca>
//      
//      This program is free software; you can redistribute it and/or modify
//      it under the terms of the GNU General Public License as published by
//      the Free Software Foundation; either version 2 of the License, or
//      (at your option) any later version.
//      
//      This program is distributed in the hope that it will be useful,
//      but WITHOUT ANY WARRANTY; without even the implied warranty of
//      MERCHANTABILITY or FITNESS FOR A PARTICULAR PURPOSE.  See the
//      GNU General Public License for more details.
//      
//      You should have received a copy of the GNU General Public License
//      along with this program; if not, write to the Free Software
//      Foundation, Inc., 51 Franklin Street, Fifth Floor, Boston,
//      MA 02110-1301, USA.


#include <iostream>
#include "itkImage.h"
#include "itkVector.h"
#include "itkImageFileReader.h"
#include "itkImageFileWriter.h"
#include "itkWarpImageFilter.h"
#include "itkImageRegionIteratorWithIndex.h"
#include <itkBSplineInterpolateImageFunction.h>


int main(int argc, char** argv)
{
	if (argc != 6){
		std::cout<<"Improper usage."<<std::endl;
		std::cout<<"Usage:"<<std::endl;
		std::cout<<argv[0]<<" [Input Image Filename] [Output Image Filename] [Output Deformation Image] [Strain Field Type] [Maximum Strain in %]"<<std::endl<<std::endl;
		std::cout<<"Strain Field Type:\n 1 = Constant \n 2 = Linear \n 3 = Sinusoidal"<<std::endl<<std::endl;
		return EXIT_FAILURE;
	}
	
	// Define the images
	typedef short										ImagePixelType;
	typedef float										DeformationPixelComponentType;
	const unsigned int dimension = 3;
	typedef itk::Vector<DeformationPixelComponentType,dimension>	DeformationPixelType;
	typedef itk::Image<ImagePixelType,dimension>		ImageType;
	typedef	itk::Image<DeformationPixelType,dimension>	DeformationImageType;
	
	// Read the input image
	typedef	itk::ImageFileReader<ImageType>				ImageReaderType;
	ImageReaderType::Pointer	reader = ImageReaderType::New();
	reader->SetFileName( argv[1] );
	try{
		reader->Update();
	}
	catch( itk::ExceptionObject &err ){
		std::cout<<"Error when reading the input image.  Error message:"<<std::endl;
		std::cout<<err<<std::endl<<std::endl;
		return EXIT_FAILURE;
	}
	
	// Create the deformation image from the input image
	DeformationImageType::Pointer	deformImage = DeformationImageType::New();
	deformImage->SetOrigin( reader->GetOutput()->GetOrigin() );
	deformImage->SetSpacing( reader->GetOutput()->GetSpacing() );
	deformImage->SetRegions( reader->GetOutput()->GetLargestPossibleRegion() );
	deformImage->Allocate();
	
	// Fill in the deformation image
	typedef itk::ImageRegionIteratorWithIndex< DeformationImageType >	DeformationImageIteratorType;
	DeformationImageIteratorType	deformIt( deformImage, deformImage->GetLargestPossibleRegion() );
	float maxStrain = atof( argv[5] )/100;
	unsigned int strainType = atoi( argv[4] );
	DeformationImageType::SpacingType	spacing = deformImage->GetSpacing();
	DeformationImageType::SizeType		size = deformImage->GetLargestPossibleRegion().GetSize();
	
	switch ( strainType ){
		case 1:{ // constant
			for (deformIt.GoToBegin(); ! deformIt.IsAtEnd(); ++deformIt){
				DeformationPixelType			currentPixel;
				DeformationImageType::IndexType	currentIndex;
				currentIndex = deformIt.GetIndex();
				currentPixel[0] = 0;
				currentPixel[1] = 0;
				currentPixel[2] = maxStrain*currentIndex[2]*spacing[2]; // constant strain = linear deformation
				deformIt.Set(currentPixel);
			}
			break;
		}
		
		case 2:{ // linear
			//~ float coefficient = maxStrain/pow((size[2]*spacing[2]),2);
			float coefficient = maxStrain/(2*size[2]*spacing[2]);
			for (deformIt.GoToBegin(); ! deformIt.IsAtEnd(); ++deformIt){
				DeformationPixelType			currentPixel;
				DeformationImageType::IndexType	currentIndex;
				currentIndex = deformIt.GetIndex();
				currentPixel[0] = 0;
				currentPixel[1] = 0;
				currentPixel[2] = coefficient*pow((currentIndex[2]*spacing[2]),2); // linear strain = quadratic deformation
				deformIt.Set(currentPixel);
			}
			break;
		}
			
		case 3:{ // sinusoidial
			float amplitude = maxStrain*(size[2]*spacing[2])/3.141592654;
			float period = 3.141592654/size[2];	
			for (deformIt.GoToBegin(); !deformIt.IsAtEnd(); ++deformIt){
				DeformationPixelType			currentPixel;
				DeformationImageType::IndexType	currentIndex;
				currentIndex = deformIt.GetIndex();
				currentPixel[0] = 0;
				currentPixel[1] = 0;
				currentPixel[2] = -amplitude*cos( currentIndex[2]*period );			
				deformIt.Set(currentPixel);
			}
			break;
		}
		
		default :
			std::cout<<"No valid strain field type chosen."<<std::endl;
			std::cout<<"The given value was: "<< argv[4]<<std::endl;
			std::cout<<"Run: "<<argv[0]<<" with no options for more information."<<std::endl;
			return EXIT_FAILURE;
	}
	
	typedef itk::BSplineInterpolateImageFunction< ImageType, double, double> InterpolatorType;
	InterpolatorType::Pointer	interpolator = InterpolatorType::New();
	interpolator->SetSplineOrder( 4 );
	
	// create the deformation filter
	typedef itk::WarpImageFilter< ImageType, ImageType, DeformationImageType >	WarpImageFilterType;
	WarpImageFilterType::Pointer	warpFilter = WarpImageFilterType::New();
	warpFilter->SetInterpolator( interpolator );
	warpFilter->SetInput( reader->GetOutput() );
	warpFilter->SetDeformationField( deformImage );
	warpFilter->SetOutputSpacing( reader->GetOutput()->GetSpacing() );
	warpFilter->SetOutputOrigin( reader->GetOutput()->GetOrigin() );
	warpFilter->SetNumberOfThreads( 4 );
	try{
		warpFilter->Update();
	}
	catch( itk::ExceptionObject &err){
		std::cout<<"Error updating the warping filter.  Error message:"<<std::endl;
		std::cout<<err<<std::endl<<std::endl;
		return EXIT_FAILURE;
	}
	
	// Write the image
	typedef itk::ImageFileWriter< ImageType >		WriterType;
	WriterType::Pointer		writer = WriterType::New();
	writer->SetFileName( argv[2] );
	writer->SetInput( warpFilter->GetOutput() );
	try{
		writer->Update();
	}
	catch( itk::ExceptionObject &err ){
		std::cout<<"Error when writing the output image.  Error message:"<<std::endl;
		std::cout<<err<<std::endl<<std::endl;
		return EXIT_FAILURE;
	}
	
	// Write the deformation field
	typedef itk::ImageFileWriter< DeformationImageType >	DeformWriterType;
	DeformWriterType::Pointer	deformWriter = DeformWriterType::New();
	deformWriter->SetFileName( argv[3] );
	deformWriter->SetInput( deformImage );
	try{
		deformWriter->Update();
	}
	catch( itk::ExceptionObject &err ){
		std::cout<<"Error when writing the deformation image.  Error message:"<<std::endl;
		std::cout<<err<<std::endl<<std::endl;
		return EXIT_FAILURE;
	}
	
	
	return 0;
}

\end{lstlisting}